\documentclass[a4paper]{article}

\usepackage[utf8]{inputenc}
\usepackage{amsmath,amssymb,enumerate,fullpage,float}

%\usepackage[usenames,dvipsnames,pdf]{pstricks}
\usepackage{epsfig,microtype}
\usepackage{hyperref}

\title{General Facts}

\pagestyle{plain}
\setlength{\parindent}{0cm}

% \usepackage[ngerman]{babel}
%\usepackage[T1]{fontenc} % Trennung bei Woertern mit Umlauten
\usepackage{amsthm,cite}
\usepackage{url}
\usepackage{graphicx}
\usepackage{comment,microtype}
\usepackage{stmaryrd,faktor}

%\usepackage{pst-all}

%\usepackage{tikz-cd} % for commutative diagrams, but didn't work here
\usepackage{amscd} % for (very simple) commutative diagrams
\usepackage[all]{xy}

%\hyphenation{mani-fold}
%\hyphenation{sub-mani-fold}

\newcommand{\RR}{\mathbb{R}}
\newcommand{\CC}{\mathbb{C}}
\newcommand{\ZZ}{\mathbb{Z}}
\newcommand{\NN}{\mathbb{N}}
\newcommand{\QQ}{\mathbb{Q}}
%\newcommand{\K}{\mathbb{K}}
\newcommand{\PP}{\mathbb{P}}
%\newcommand{\CP}{\mathbb{CP}}
%\newcommand{\RP}{\mathbb{RP}}
\newcommand {\kk} {\Bbbk}
\newcommand{\VV}{\mathbb V}
\newcommand{\HH}{\mathbb{H}}
\newcommand{\KK}{\mathbb{K}}


\newcommand{\mfd}{\mathfrak d}
\newcommand{\mfD}{\mathfrak D}
\newcommand{\mfc}{\mathfrak c}
\newcommand{\mfU}{\mathfrak U}
\newcommand{\mfu}{\mathfrak u}
\newcommand{\mfB}{\mathfrak B}
\newcommand{\mft}{\mathfrak t}
\newcommand{\mfm}{\mathfrak m}
\newcommand{\mfb}{\mathfrak b}
\newcommand{\mfj}{\mathfrak j}
\newcommand{\mfv}{\mathfrak v}
\newcommand{\mfM}{\mathfrak M}
\newcommand{\mfX}{\mathfrak X}
\newcommand{\mfR}{\mathfrak R}
\newcommand{\mfT}{\mathfrak T}

\newcommand{\mcM}{\mathcal M}
\newcommand{\mcC}{\mathcal C}
\newcommand{\mcN}{\mathcal N}
\newcommand{\mcX}{\mathcal X}
\newcommand{\mcR}{\mathcal R}
\newcommand{\mcO}{\mathcal O}
\newcommand{\mcA}{\mathcal A}
\newcommand{\mcF}{\mathcal F}
\newcommand{\mcE}{\mathcal E}
\newcommand{\mcU}{\mathcal U}
\newcommand{\mcJ}{\mathcal J}
\newcommand{\mcT}{\mathcal T}
\newcommand{\mcD}{\mathcal D}
\newcommand{\mcL}{\mathcal L}
\newcommand{\mcG}{\mathcal G}


\newcommand{\dd}{\mathrm{d}}
\newcommand{\dvol}{\mathrm{vol}_{\Sigma}}
\newcommand{\ddd}{{(d,d')}}
\newcommand{\ddv}{{2d+2d'-\nu}}
\newcommand{\ddvm}{{-2d-2d'-\nu}}
\newcommand{\PtP}{{\PP^1 \times \PP^1}}
\newcommand{\ii}{\mathrm{i}}
\newcommand{\si}{\Sigma}
\newcommand{\del}{\partial}
\newcommand{\db}{\overline{\partial}}
\newcommand{\cin}{C^{\infty}}
\newcommand{\lra}{\longrightarrow}
\newcommand{\ol}[1]{\overline{#1}}
\newcommand{\wt}[1]{\widetilde{#1}}
\newcommand{\ddt}{\frac{\dd}{\dd t}_{|_{t=0}}}


\newcommand{\id}{\operatorname{id}}
\newcommand{\Hom}{\operatorname{Hom}}
\newcommand{\Mult}{\operatorname{Mult}}
\newcommand{\Mono}{\operatorname{Mono}}
\newcommand{\Pic}{\operatorname{Pic}}
\newcommand{\Init}{\operatorname{Init}}
\newcommand{\Supp}{\operatorname{Supp}}
\newcommand{\Sing}{\operatorname{Sing}}
\newcommand{\Trees}{\operatorname{Trees}}
\newcommand{\Spec}{\operatorname{Spec}}
\newcommand{\Spf}{\operatorname{Spf}}
\newcommand{\Val}{\operatorname{Val}}
\newcommand{\Int}{\operatorname{Int}}
\newcommand{\Asym}{\operatorname{Asym}}
\newcommand{\Interstices}{\operatorname{Interstices}}
\newcommand{\Joints}{\operatorname{Joints}}
\newcommand{\Opp}{\operatorname{Opp}}
\newcommand{\dlog}{\operatorname{dlog}}
\newcommand{\Aut}{\operatorname{Aut}}
\newcommand{\Log}{\operatorname{Log}}
\newcommand{\ts}{\mathrm{T}}
\newcommand{\ind}{\operatorname{index}}
%\newcommand{\dim}{\operatorname{dim}}
%\newcommand{\ker}{\operatorname{ker}}
\newcommand{\coker}{\operatorname{coker}}
\newcommand{\End}{\operatorname{End}}
%\newcommand{\Aut}{\operatorname{Aut}}
\newcommand{\image}{\operatorname{Im}}
\newcommand{\grapho}{\operatorname{graph}}
\newcommand{\Trace}{\operatorname{Tr}}
\newcommand{\ver}{\operatorname{ver}}
\newcommand{\area}{\operatorname{area}}
\newcommand{\diff}{\operatorname{Diff}}
\newcommand{\arc}{\operatorname{arc}}
\newcommand{\Ham}{\operatorname{Ham}}

\renewcommand{\Re}{\operatorname{Re}}
\renewcommand{\Im}{\operatorname{Im}}


\newcommand{\out}{\mathrm{out}}
\newcommand{\disk}{\mathrm{disk}}
\newcommand{\pt}{\mathrm{pt}}
\newcommand{\trop}{\mathrm{trop}}
\newcommand{\prim}{\mathrm{prim}}
\newcommand{\ev}{\mathrm{ev}}
\newcommand{\GL}{\mathrm{GL}}


\theoremstyle{definition}
\newtheorem{thm}{Theorem}
\newtheorem{prop}{Proposition}
\newtheorem{lem}{Lemma}
\newtheorem{cor}{Corollary}
\newtheorem{conjecture}{Conjecture}

\theoremstyle{definition}
\newtheorem{definition}{Definition}

\theoremstyle{remark}
\newtheorem{rmk}{Remark}

\theoremstyle{remark}
\newtheorem{choice}{Choice}

\theoremstyle{remark}
\newtheorem{ex}{Example}

\everymath{\displaystyle}

\begin{document}

\maketitle 

\tableofcontents

\section{Algebra}

\begin{definition}[Novikov Ring]
  The Novikov ring over a base field $\KK$ is defined by
  \begin{equation*}
    \Lambda_0:=\left\{\sum_{i=0}^{\infty}a_iT^{\lambda_i}\middle|a_i\in\KK,\lambda_i\in\RR_{\geq 0},\lim_{i\to\infty}\lambda_i=+\infty\right\}.
  \end{equation*}
  The Novikov field is the field of fractions of $\Lambda_o$, i.e.\ 
  \begin{equation*}
    \Lambda:=\left\{\sum_{i=0}^{\infty}a_iT^{\lambda_i}\middle|a_i\in\KK,\lambda_i\in\RR,\lim_{i\to\infty}\lambda_i=+\infty\right\}.
  \end{equation*}
\end{definition}

\section{Analysis}

\subsection{Basics}

\begin{definition}[Equicontinuos Family]
  Let $X$ be a compact Hausdorff space and $C(X)$ denote the space of real-valued continous funtions on $X$. A subset $F\subset C(X)$ is called \emph{equicontinous} if for every $x\in X$ and every $\epsilon>0$ there exists a neighborhood $U$ of $x$ such that for all $y\in U$ and all $f\in F$ we have $|f(y)-f(x)|<\epsilon$.
\end{definition}

\begin{definition}[Pointwise Bounded Family]
  A set $F\subset C(X)$ of continous real-valued functions on some compact Hausdorff space $X$ is \emph{pointwise bounded} if for every $x\in X$ we have $\sup\{|f(x)|:f\in F\}<\infty$.
\end{definition}

\begin{thm}[Arz\'ela--Ascoli]
  Let $X$ be a compact Hausdorff space. Then $F\subset C(X)$ is relatively compact in the topology induced by uniform norm if and only if it is equicontinous and pointwise bounded.
\end{thm}

\begin{cor}
  Consider a sequence of real-valued continous functions $\{f_n\}_{n\in\NN}$ defined on a closed and bounded interval $[a,b]$ of the real line. There exists a subsequence of $\{f_n\}$ which converges uniformly if and only if this sequence is uniformly bounded and equicontinous.
\end{cor}

\subsection{Functional Analysis}

\begin{definition}[Weak Topologies]
  Let $X$ be a topological vector space. Then the \emph{weak} topology on $X$ is defined by $x_n\lra x$ if and only if $\phi(x_n)\lra\phi(x)\;\forall\phi\in X^*$, where $X^*$ is the topological dual of $X$, i.e.\ the space of all continuous linear functionals on $X$. The \emph{weak-$\star$ topology} on $X^*$ is defined by $\phi_n\lra\phi$ if and only if $\phi_n(x)\lra\phi(x)\;\forall x\in X$.
\end{definition}

\begin{rmk}
  The weak-$\star$ topology on $X^*$ is weaker than the weak topology on $X^*$, because in general $X\lra X^{**}$ is an injective linear map and the weak-$\star$ topology is defined as the coarsest topology such that the image of $X$ in $X^{**}$ still consists of continuous maps $X^*\lra\RR$.
\end{rmk}

\begin{thm}[Open Mapping Theorem]
  Let $X$ and $Y$ be Banach (or Fr\'echet) spaces and $A:X\lra Y$ a surjective continous linear operator. Then $A$ is an open map.
\end{thm}

\begin{thm}[Bounded Inverse Theorem]
  If $A:X\lra Y$ is a bijective continous linear operator between the Banach spaces $X$ and $Y$, then the inverse operator $A^{-1}:Y\lra X$ is continous as well.
\end{thm}

\begin{thm}[Closed Graph Theorem]
   If $A:X\lra Y$ is a linear operator between the Banach spaces $X$ and $Y$, and if for every sequence $(x_n)$ in $X$ with $x_n\lra 0$ and $Ax_n\lra 0$ it follows that $y=0$, then $A$ is continous.
\end{thm}

\begin{definition}[Fr\'echet Space]
  A topological vector space $X$ is called a \emph{Fr\'echet} space if and only if it satisfies one of the following equivalent triples of conditions:
  \begin{enumerate}
    \item $X$ is locally convex, its topology can be induced by a translation invariant metric and it is a complete metric space.
    \item $X$ is a Hausdorff space, its topology may be induced by a countable family of semi-norms and it is complete with respect to the family of semi-norms.
  \end{enumerate}
\end{definition}

\begin{definition}[Baire Space]
  A \emph{Baire space} is a topological space with the property that for each countable collection of open dense sets their intersection is also dense.
\end{definition}

\begin{thm}[Baire Category Theorem]
  \begin{enumerate}
    \item Every complete metric space is Baire.
    \item Every locally compact Hausdorff space is Baire.
    \item A non-empty complete metric space is not the countable union of nowhere-dense closed sets.
  \end{enumerate}
\end{thm}

\begin{definition}[Comeagre or Residual Set]
  
\end{definition}

\begin{definition}[Compact Operator]
  
\end{definition}

\subsection{Fixed Point Theorems}

\begin{thm}[Brouwer Fixed Point Theorem]
  Every continous function from a convex compact subset $K$ of a Euclidean space to $K$ itself has a fixed point.
\end{thm}

\begin{thm}[Schauder Fixed Point Theorem]
  Every continous function from a convex compact subest $K$ of a Banach space to $K$ itself has a fixed point.
\end{thm}

\begin{definition}[Contraction Mapping]
  Let $(X,d)$ be a metric space. Then a map $T:X\lra X$ is called a contraction mapping on $X$ if there exists a $q\in[0,1]$ such that $d(T(x),T(y))\leq qd(x,y)$ for all $x,y\in X$.
\end{definition}

\begin{thm}[Banach Fixed Point Theorem]
  Let $(X,d)$ be an non-empty complete metric space with a contraction mapping $T:X\lra X$. Then $T$ admits a unique fixed-point $x^*\in X$. Furthermore, $x^*$ can be found as follows: start with an arbitrary element $x_0\in X$ and define a sequence $\{x_n\}$ by $x_{n+1}=T(x_{n-1})$, then $x_n\lra x^*$.
\end{thm}

\begin{definition}[Lefschetz Number]
  Let $f:X\lra X$ be a continous map from a compact triangulizable space $X$ to itself. Define the \emph{Lefschetz number} $\Lambda_f$ by
  \begin{equation*}
    \Lambda_f:=\sum_{k\geq k}(-1)^k\Trace(f_*|_{H_k(X,\QQ)})
  \end{equation*}
\end{definition}

\begin{thm}[Lefschetz Fixed Point Theorem]
  If $\Lambda_f\neq 0$ then $f$ has at least one fixed point. Furthermore, if you denote by $i(f,x)$ the index of the fixed point $x$ and if $f$ has only finitely many fixed points, then
  \begin{equation*}
    \sum_{x\in\text{Fix}(f)}i(f,x)=\Lambda_f.
  \end{equation*}
\end{thm}

\subsection{Fredholm Theory}

\begin{definition}[Fredholm Operator]
  
\end{definition}

\begin{prop}[Properties of Fredholm Operators]
  
\end{prop}

\begin{thm}[Elliptic Regularity]
  
\end{thm}

\subsection{Sobolev Spaces}

\begin{definition}[Sobolev Spaces]
  
\end{definition}

\begin{thm}[Sobolev Embedding Theorems]
  
\end{thm}

\section{Topology}

\begin{rmk}
  A topology $\tau_1$ is called weaker or coarser than $\tau_2$ if $\tau_1$ contains less open sets than $\tau_2$. If $\tau_2\subset\tau_1$ then $\tau_1$ is called stronger or finer. This means that if a sequence converges in one topology then it also converges in every weaker topology as there are less open sets to test the condition on.
\end{rmk}

\section{Algebraic Topology}

\begin{thm}[Alexander Duality]
  
\end{thm}

\section{Differential Topology}

\begin{definition}[Ruled $4$-Manifold]
  A manifold $M$ of dimension $4$ is called \emph{ruled} if it is a $S^2$-bundle over a closed Riemann surface.
\end{definition}

\begin{definition}[Fibered Knot]
  A knot $K\subset S^3$ is called \emph{fibered} if there exists a $S^1$-family $F_t$ with $t\in S^1$ of Seifert surfaces for $K$ such that $F_s\cap F_t=K$ for all $s\neq t$.
\end{definition}

\begin{prop}
  A knot is fibered if and only if it is the binding of some open book decomposition of $S^3$.
\end{prop}

\begin{definition}[Heegard Splittings and Diagrams]
  
\end{definition}

\section{Riemannian Geometry}

\subsection{Hypersurfaces}

\begin{definition}[Shape operator or Weingarten map]
  Let $S\subset\RR^n$ be a smooth hypersurface in Euclidean $n$-space. Then the \emph{shape operator} or \emph{Weingarten map} $S_p$ is defined by
  \begin{equation*}
    \langle S_p(v),w\rangle=\langle \dd\nu(v),w\rangle
  \end{equation*}
  for all $v,w\in\ts_pS$, where $\nu:S\lra S^{n-1}$ is the Gauss map, i.e.\ it is given by $\nu(p)=N_p$, where $N_p$ is a normal vector to $S$ at $p$.
\end{definition}

\subsection{Hyperbolic Geometry}

\begin{definition}[Geodesic lamination]
  A \emph{geodesic lamination} on a complete hyperbolic surface $S$ is a closed subset of $S$ foliated by complete simple geodesics.  
\end{definition}

\begin{definition}[Transversal measures on laminations]
  A \emph{transversal measure} on a lamination $\lambda$ is a measure on the collection of arcs on $S$ transversal to $\lambda$ which  is invariant under isotopies of $S$ preserving $\lambda$. A \emph{measured lamination} is a pair of a lamination and a transversal measure.
\end{definition}

\begin{rmk}
  Let $\wt{S}_{\infty}$ be the boundary at infinity of $\HH^2$. Then $\mcG\mcL(S)$ denotes the subset of closed subsets of $\faktor{\wt{S}_{\infty}\times \wt{S}_{\infty}}{\sim}$ parametrizing geodesics (and thus laminations) with the Hausdorff topology. It is compact. A geodesic lamination is called \emph{minimal} if every leaf is dense. See \cite{Brock2014}.
\end{rmk}
\section{Riemann Surfaces}

\section{Fibre Bundles}

\subsection{Definitions}

\subsection{Existence}


\begin{lem}[Ehresmann's lemma]
  Let $M$ and $N$ be smooth manifolds and $f:M\lra N$ a smooth map. If $f$ is a proper surjective submersion then $f$ is a locally trivial fibration.
\end{lem}

\section{Symplectic Geometry}

\subsection{Basics}

\begin{definition}[Hamiltonian Diffeomorphism]
  A Hamiltonian diffeomorphism $\Psi\in\operatorname{Ham}(M,\omega)$ of a symplectic manifold is a time-one map of a time-dependent Hamiltonian flow.
\end{definition}

\subsection{Examples}

\subsection{Lagrangian Submanifolds}

\begin{definition}[Properties of Lagrangians]
Let $L\subset (M,\omega)$ be a Lagrangian submanifold. We call $L$
\begin{itemize}
  \item monotone, if there exists a $\tau>0$ such that $\omega=\tau\mu$, where $\omega:\pi_2(M,L)\lra\RR$ is the symplectic form as a map on $\pi_2(M,L)$ and $\mu:\pi_2(M,L)\lra\ZZ$ is the Maslov index,
  \item exact, if $\omega=\dd\lambda$ is exact and if $[\lambda|_L]=0\in H^1(L)$,
  \item displacable, if there exists $\Psi\in\operatorname{Ham}(M,\omega)$ such that $\Psi(L)\cap L=\emptyset$,
  \item semi-monotone, if .
\end{itemize}
\end{definition}

\begin{definition}[Lagrangian Cobordism]
  Let $(M,\omega)$ be a symplectic manifold, denote by $\pi:\RR^2\times M\lra\RR^2$ the projection and equip $\RR^2$ with the standard symplectic structure. A \emph{Lagrangian cobordism} $V:(L_j')\leadsto(L_i)$ between two families of closed Lagrangian submanifolds $(L_i)_{1\leq i\leq k_-}$ and $(L_j')_{1\leq j\leq k_+}$  is a Lagrangian embedding $V\subset [0,1]\times\RR\times M$ such that for some $\epsilon>0$ we have
  \begin{align*}
    V\cap\pi^{-1}([0,\epsilon)\times \RR)&=\coprod_i([0,\epsilon)\times\{i\})\times L_i \\
    V\cap\pi^{-1}([1-\epsilon,1)\times \RR)&=\coprod_j([1-\epsilon,1)\times\{j\})\times L_j'
  \end{align*}
\end{definition}

\begin{definition}[Fukaya Category]
  Let $(M,\omega)$ be a symplectic manifold with $2c_1(\ts M)=0$. The objects of the compact Fukaya category $\mcF(M,\omega)$ are compact, closed, oriented, spin Lagrangian submanifolds $L\subset M$ such that $[\omega]|_{\pi_2(M,L)}=0$ and vanishing Maslov class $\mu_L=0\in H^1(L,\ZZ)$ together with the choice of a spin structure and a graded lift of $L$.

For every pair of objects $(L,L')$ we choose perturbation data $H_{L,L'}\in\cin([0,1]\times M,\RR)$ and $J_{L,L'}\in\cin([0,1],\mcJ(M,\omega))$ and for all tuples of objects $(L_0,\ldots,L_k)$ and all moduli spaces of discs we choose consistent perturbation data $(H,J)$ compatible with the choices made for the pairs of objects $(L_i,L_j)$ such that we have transversality for alle moduli spaces of perturbed $J$-holomorphic discs.

We set $\hom(L,L'):=CF(L,L';H_{L,L'},J_{L,L'})$ and the differential $\mu^1$ and composition $\mu^2$ and higher operations $\mu^k$ are given by counts of perturbed $J$-holomorphic discs with boundary on the $k$ arguments. This makes $\mcF(M,\omega)$ a $\Lambda$-linear, $\ZZ$-graded, non-unital (but cohomologically unital) $A_{\infty}$-category.
\end{definition}

\begin{definition}[Lagrangian Suspension Construction]
  
\end{definition}

\begin{definition}[Lagrangian Surgery]
  
\end{definition}

\begin{definition}[Symplectic Folding]
  
\end{definition}

\begin{definition}[Lagrangian Correspondence]
  
\end{definition}

\begin{ex}[Lagrangian Correspondences]
  
\end{ex}

\subsection{Various Topics}

\begin{definition}[Symplectic and Stein Cobordism, \cite{Etnyre2002}]
  A contact $3$-manifold $(M_1,\xi_1)$ is \emph{symplectically} (resp.\ \emph{Stein}) cobordant to $(M_2,\xi_2)$ if there exists a symplectic (resp.\ Stein) $4$-manifold $(X,\omega)$ with $\del X=M_2-M_1$ and a vector field $V$ defined on a neighborhood of $M_1\cup M_2\subset X$ for which $\mcL_V\omega=\omega, V\pitchfork M_1\cup M_2$ and the normal orientation of $M_1\cup M_2$ agrees with $V$.
\end{definition}

\begin{rmk}
Symplectic and Stein cobordisms are not a symmetric relation, see \cite{Etnyre2002}.
\end{rmk}

\begin{definition}[Weinstein manifold]
A \emph{Weinstein manifold} is a tuple $(V,\omega,X,\phi)$, where $(V,\omega)$ is a symplectic manifold , $\phi:V\lra\RR$ is an exhausting Morse funtion and $X$ is a complete Liouville vector field which is gradient-like for $\phi$.
\end{definition}

\begin{rmk}
  A function $\phi:V\lra\RR$ is called \emph{exhausting} if it is proper and bounded from below.
\end{rmk}

\begin{definition}[Stein manifold]
  The following statements are equivalent for a non-compact complex manifold $(V,J)$:
  \begin{enumerate}
    \item $(V,J)$ admits a proper holomorphic embedding into some $\CC^N$
    \item $V$ admits an exhausting $J$-convex function $\phi:V\lra\RR$
    \item $V$ is holomorphically convex, $\forall x\in V\exists f_1,\ldots,f_n:V\lra\CC$ holomorphic such that they form a holomorphic coordinate system at $x$ and $\forall x\neq y\in V\exists f:V\lra\CC$ holomorphic s.t.\ $f(x)\neq f(y)$.
  \end{enumerate}
  Any complex manifold satisfying one (and thus all) of the above is called a \emph{Stein manifold}.
\end{definition}

\begin{definition}[Liouville Domain]
  Let $(M,\theta)$ be a compact manifold with boundarz and $\theta\in\Omega^1(M)$ with $\dd\theta$ symplectic and $\theta=r\alpha$ close to the boundary, where we identify a neighborhood of the boundary with $[1-\epsilon,1]\times\del M$, $r$ is the coordinate in the first factor and $\alpha$ is a contact form on the boundary. Such a $(M\theta)$ is called a \emph{Liouville domain}.
\end{definition}

\begin{definition}[Open Book Decomposition]
  An open book $(K,\theta)$ of a manifold $V^{2n+1}$ consists of a submanifold $K\subset V$ of codimension $2$ with trivial normal bundle and a fibration $\theta:V\setminus K\lra S^1$ which on a neighborhood $K\times D^2$ of $K\times\{0\}$ is given by the angle coordinate on $D^2$.
\end{definition}

\begin{definition}
  A contact structure $\xi$ on a manifold $V^{2n+1}$ is \emph{supported} by an open book $(K,\theta)$ if there exists a contact form such that $\xi=\ker\alpha$ and if
  \begin{enumerate}
    \item $\alpha$ induces a contact form on $K$,
    \item $\dd\alpha$ induces a symplectic form on each fibre $F$ of $\theta$ and
    \item orientation of $K$ induced by $\alpha$ equals the orientation of $K$ as the boundary of $(F,\dd\alpha)$.
  \end{enumerate}
\end{definition}

\begin{rmk}
  $K$ is called the \emph{binding},  $\ol{F}^V$ is called the \emph{page}.
\end{rmk}

\begin{thm}[Giroux]
  Every contact manifold is supported by by an open book whose fibres are Weinstein.
\end{thm}

\begin{thm}[Giroux Correspondence]
  If $M$ is a closed oriented $3$-manifold there is a one-to-one correspondence between 
  \begin{gather*}
    \{\text{oriented contact structures on $M$ up to isotopy}\} \\
    \text{and} \\
    \{\text{open book decompositions of $M$ up to positive stabilization}\}.
  \end{gather*}
\end{thm}

\begin{definition}[Positive Stabilization]
  
\end{definition}

\begin{definition}[Lefschetz Fibration]
  Let $(V,\omega)$ be a symplectic manifold. A topological Lefschetz fibration is a tuple $(A,\{x_{\alpha}\},f)$, where $A\subset V$ is a codimension-2 symplectic submanifold, $x_{\alpha}\in V\setminus A$ are finitely many points in $V$ and $f:V\setminus A\lra S^2$ is a submersion on $V\setminus(A\cup\{x_{\alpha}\}$ and $f(x_{\alpha})\neq f(x_{\beta})$ for all $\alpha\neq\beta$ which satisfy the following:
  \begin{enumerate}
    \item at each $a\in A$ there exist local compatible complex coordinates $z_i$ such that $A$ is locally defined by $z_1=z_2=0$ and $f$ is given locally by $(z_1,\ldots,z_n)\longmapsto\frac{z_1}{z_2}\in\CC P^1\cong S^2$ and
    \item at a point $x_{\alpha}$ there exist local compatible complex coordinates $z_i$ such that $f$ is given locally by $(z_1,\ldots,z_n)\longmapsto f(x_{\alpha})+z_1^2+\cdots+z_n^2$.
  \end{enumerate}
\end{definition}

\begin{rmk}
  A system of local complex coordinates $(z_1,\ldots,z_n)$ on a compact symplectic manifold of dimension $2n$ is called compatible if $\omega$ is in those coordinates a positive form of type $(1,1)$ at the origin.
\end{rmk}

\begin{definition}[Thimbles or Vanishing Spheres]
  
\end{definition}

\begin{thm}[Lefschetz]
  Suppose $(V,\omega)$ is a compact symplectic manifold such that $[\omega]\in H^2(V;\ZZ)$. For a sufficiently large integer $k\in\NN$ there is a topological Lefschetz pencil on $V$ whose fibres are symplectic (outside the singularities) and homologous to $k$ times the Poincar\'e dual of $[\omega]$.
\end{thm}

\begin{thm}[Lefschetz Hyperplane Theorem]
  \begin{enumerate}
    \item If $M\subset\CC^N$ is a non-singular affine algebraic variety with real dimension $2k$ then $H_i(M;\ZZ)\cong 0$ for $i>k$.
    \item
  \end{enumerate}
\end{thm}

\begin{definition}[Symplectically Aspherical]
  A symplectic manifold $(M,\omega)$ is called \emph{symplectically aspherical} if for any smooth map $f:S^2\lra M$ one has $\int_{S^2}f^*M=0$ or equivalently $\omega|_{\pi_2(M)}=0$ or $[\omega]|_{\operatorname{im}(\text{hur}_2)}=0$, where $\text{hur}_2$ denotes the Hurewicz homomorphism $\pi_2(M)\lra H_2(M)$.
\end{definition}

\begin{definition}[Stable Hamiltonian Structure]
  
\end{definition}

\begin{rmk}
  Relations of stable Hamiltonian structures to other things
\end{rmk}

\begin{definition}[Dehn Twist]
  
\end{definition}

\begin{definition}[Asymptotic Operators]
  
\end{definition}

\subsection{$J$-holomoprhic Curves}

\begin{definition}[Properties of $J$-holomorphic Curves]
A $J$-holomorphic curve $u:\si\lra X$ is called
  \begin{enumerate}[(i)]
    \item \emph{simple}
    \item \emph{somewhere injective}
    \item \emph{multiply covered}
  \end{enumerate}
\end{definition}

\begin{rmk}
  Relation between properties.
\end{rmk}

\begin{thm}[Micaleff--White]
  
\end{thm}

\begin{ex}[Lantern Example]
  
\end{ex}

\begin{thm}[Automatic Transversality]
  
\end{thm}

\subsection{Compactness Results}

\begin{thm}[Compactness of Morse Gradient Flow Lines]
  
\end{thm}

\begin{thm}[Gromov Compactness]
  
\end{thm}

\begin{thm}[SFT Compactness]
  
\end{thm}

\subsection{Conjectures}

\begin{conjecture}[Conley's Conjecture]
  A Hamiltonian diffeomorphism of a suitable (e.g.\ surface, torus, closed symplectically aspherical or cotangent bundle) symplectic manifold has infinitely many simple periodic points.
\end{conjecture}

\begin{conjecture}[Weinstein Conjecture]
  If $M$ is a closed oriented odd-dimensional manifold with a contact form $\lambda$ then the associated Reeb vector field has a closed orbit.
\end{conjecture}

\begin{conjecture}[Arnold Conjecture]
  A Hamiltonian diffeomorphism on a symplectic manifold $M$ has at least as many fixed points as the minimal number of critical points of a Morse function on $M$.
\end{conjecture}

\begin{conjecture}[Arnold--Givental Conjecture]
  
\end{conjecture}

\subsection{Dynamics}

\begin{definition}[Types of Orbits]
A periodic orbit $\gamma:S^1\lra M$ of a flow $\phi_t$ is called
\begin{itemize}
  \item nondegenerate, if the linearized flow after one period on a transversal space $\Psi:V\lra V$ (with $V\subset \ts_p M$ such that $V\pitchfork \RR\ddt\phi_t(p)$) has no eigenvalue equal to $1$, or equivalently $\det(\Psi-\id)\neq 0$,
  \item elliptic, if every eigenvalue of $\Psi$ is in the unit circle,
  \item hyperbolic, if every eigenvalue of $\Psi$ has norm diferrent from $1$,
  \item (un-)stable, if every eigenvalue of $\Psi$ has norm (bigger) smaller than $1$.
\end{itemize}
\end{definition}

\begin{definition}[Hofer's Metric]
  Let $(m,\omega)$ be a connected symplectic manifold without boundary. Denote by $\Ham^c(M,\omega)$ all Hamiltonian diffeomorphisms with compact support. Given a path $\{\phi_t\}_{0\leq t\leq 1}\subset\Ham^c(M,\omega)$ and a family of Hamiltonian functions $\{H_t\}$ generating this flow we define
  \begin{equation*}
    \mcL(\{\phi_t\}):=\int_0^1\left(\sup_{z\in M}H_t(z)-\inf_{z\in M} H_t(z)\right)\dd t.
  \end{equation*}
  Define the \emph{Hofer metric} on $\Ham^c(M,\omega)$ by
  \begin{equation*}
    \rho(\phi,\psi):=\inf_{\substack{\{\phi_t\}\subset\Ham^c(M,\omega)\\ \phi_0=\phi, \phi_1=\psi}}\mcL(\{\phi_t\}).
  \end{equation*}
\end{definition}

\begin{definition}[Bad Orbits]
  
\end{definition}

\begin{definition}[Conley--Zehnder Index]
  
\end{definition}

\begin{prop}[Properties of Conley--Zehnder Index]
  
\end{prop}

\subsection{Contact Geometry}

\begin{definition}[Contact Embedding]
  
\end{definition}

\begin{definition}[Symplectic and Stein Fillings]
  
\end{definition}

\subsection{Theorems}

\begin{thm}[Floer--McDuff--Eliashberg]
  
\end{thm}

\subsection{Homology Theories}

\subsubsection{Quantum Homology}

\subsubsection{Contact Homology}

\subsubsection{Cylindrical Contact Homology}

\subsubsection{Linearized Contact Homology}

\subsubsection{Embedded Contact Homology}

\subsubsection{Symplectic Homology}

\subsubsection{$S^1$-equivariant Symplectic Homology}

\subsubsection{Hamiltonian Floer Homology}

\subsubsection{Lagrangian Floer Homology}

\subsubsection{Rabinowitz--Floer Homology}

\subsubsection{Knot-Contact Homology}

\subsubsection{Instanton--Floer Homology}

\subsubsection{Khovanov Homology}

\subsubsection{Heegard--Floer Homology}

\bibliography{promotion}{}
\bibliographystyle{plain}

\end{document}